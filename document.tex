\documentclass[11pt,a4paper]{article}

%-------------------------------------------------------------------------------------
% Informations générales du document
%-------------------------------------------------------------------------------------

\def\myauthor{Leart Demiri}
\def\myemail{leart.dmr2@eduge.ch}
\def\mytitle{Documentation du code source}
\def\mysubtitle{PDF C\# MonogameController}

\date{\today}

%-------------------------------------------------------------------------------------
% Packages essentiels
%-------------------------------------------------------------------------------------

\usepackage[utf8]{inputenc}
\usepackage[T1]{fontenc}
\usepackage[french]{babel}
\usepackage[fr-FR]{datetime2}
\usepackage{lmodern}
\usepackage[urlcolor=blue,linkcolor=blue,colorlinks=true]{hyperref}
\usepackage{listings}
\usepackage{color,xcolor}
\usepackage{graphicx}
\usepackage{parskip}
\usepackage{fancyvrb}
\usepackage{varioref}

%-------------------------------------------------------------------------------------
% Police Helvetica
%-------------------------------------------------------------------------------------

\usepackage{helvet}
\renewcommand{\familydefault}{\sfdefault}

%-------------------------------------------------------------------------------------
% Listings C#
%-------------------------------------------------------------------------------------

\input{inc_listings_csharp}

%-------------------------------------------------------------------------------------
% Mise en page
%-------------------------------------------------------------------------------------

\usepackage[left=2.00cm, right=2.00cm, top=2.00cm, bottom=2.00cm]{geometry}

\DeclareGraphicsExtensions{.png, .PNG, .jpg, .JPG}
\graphicspath{{./images/}}

%-------------------------------------------------------------------------------------
% En-têtes et pieds de page
%-------------------------------------------------------------------------------------

\usepackage{lastpage}
\usepackage{fancyhdr}

\pagestyle{fancy}
\fancyhf{}

\fancyhead[L]{\mytitle}
\fancyhead[R]{\leftmark}

\fancyfoot[L]{\textit{\myauthor}}
\fancyfoot[C]{\tiny{-~\DTMnow~-}}
\fancyfoot[R]{\thepage~/~\pageref{LastPage}}

\renewcommand{\headrulewidth}{0.4pt}
\renewcommand{\footrulewidth}{0.4pt}

%-------------------------------------------------------------------------------------
% Début du document
%-------------------------------------------------------------------------------------

\begin{document}
	
	\tableofcontents
	\vspace{1cm}
	
	\lstlistoflistings
	
	\newpage
	
	%-------------------------------------------------------------------------------------
	% Section : Structure du projet
	%-------------------------------------------------------------------------------------
	
	\section{Structure du projet PeriphericalControl}
	
	Le projet \textbf{PeriphericalControl} s’organise autour d’une structure simple, pensée pour séparer clairement la logique d’entrée, l’affichage et les classes de configuration.  
	L’arborescence regroupe notamment les fichiers principaux C\#, les classes de gestion d’événements et les objets utilitaires utilisés par l’application.  
	Cette organisation facilite la lecture, la maintenance et l’extension du projet.
	
	%-------------------------------------------------------------------------------------
	% Section : Code source
	%-------------------------------------------------------------------------------------
	
	\section{Code source}
	
	Cette section rassemble les fichiers constituant le cœur fonctionnel du projet.  
	Ils définissent l’initialisation, la boucle principale de rendu, la gestion des périphériques et les structures de données utilisées pour représenter les entrées d’une manette.
	
	%-------------------------------------------------------------------------------------
	% Program.cs et Game1.cs
	%-------------------------------------------------------------------------------------
	
	\subsection{Program.cs}
	
	Ce fichier contient le point d’entrée de l’application.  
	
	\lstinputlisting[caption=Program.cs]{PeriphericalControl/PeriphericalControl/Program.cs}
	
	\subsection{Game1.cs}
	
	Le fichier central de l’application.  
	
	\lstinputlisting[caption=Game1.cs]{PeriphericalControl/PeriphericalControl/Game1.cs}
	
	%-------------------------------------------------------------------------------------
	% Section : Classes
	%-------------------------------------------------------------------------------------
	
	\subsection{Classes}
	
	Les classes suivantes encapsulent les structures logiques nécessaires au fonctionnement du projet.  
	Elles permettent de gérer les événements, de décrire les configurations et de manipuler les zones graphiques associées aux sticks.
	
	%---------------------------------------
	\subsubsection{InputEvent}
	
	Cette classe représente un événement d’entrée détecté par l’application.  
	Elle contient les informations nécessaires pour identifier l’action effectuée et son état.
	
	\lstinputlisting[caption=InputEvent.cs]{PeriphericalControl/PeriphericalControl/InputEvent.cs}
	
	%---------------------------------------
	\subsubsection{StickConfig}
	
	Cette classe stocke la configuration liée aux sticks analogiques.  
	Elle regroupe les valeurs de calibration ainsi que les paramètres utilisés pour déterminer l’affichage ou les limites du mouvement.
	
	\lstinputlisting[caption=StickConfig.cs]{PeriphericalControl/PeriphericalControl/StickConfig.cs}
	
	%---------------------------------------
	\subsubsection{StickSliderRects}
	
	Cette classe définit les rectangles et zones graphiques utilisés pour représenter visuellement les positions et déplacements des sticks.  
	Elle sert notamment au rendu des curseurs et des indicateurs analogiques.
	
	\lstinputlisting[caption=StickSliderRects.cs]{PeriphericalControl/PeriphericalControl/StickSliderRects.cs}
	%---------------------------------------
	\subsubsection{TestCase}
	
	Cette classe est la classe qui s'occupe de tout les test en utilisant MSTest.
	
	\lstinputlisting[caption=Tests.cs]{PeriphericalControl/PeriphericalControl/Tests.cs}
	

	
\end{document}
